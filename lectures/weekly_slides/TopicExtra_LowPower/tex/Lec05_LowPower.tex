\documentclass[]{slides}
\title{Low-power design techniques}

% Begin document
\begin{document}
%\printpdftrue % uncomment to hide pauses
% Title slide
\begin{frame} \titlepage \end{frame}

% Outline slide
%\begin{frame}{Outline} \tableofcontents \end{frame}

% ====================
% Low-power design techniques
% ====================
\begin{frame}{Power in ICs}{}
\begin{itemize}
\item Why is power important?
\pauseprint
\item We could analyse power consumption 
\begin{itemize}
\item at device level
\item at system (board) level
\item at \acf{IC} level
\item at transistor level
\end{itemize}
\pauseprint
\item We will focus our efforts at the \ac{IC} level.
\item We first need to understand, what is power?
\pauseprint
\begin{itemize}
\item The amount of energy transferred or converted per unit time.
\item Electrical instantaneous power is simply the product of the current and voltage in a device.

\end{itemize}
\end{itemize}
\begin{equation}
P = IV
\end{equation}
\end{frame}

% ====================
% Low-power design techniques
% ====================
\begin{frame}{Power in ICs}{}
\alertblue{Definitions}
\begin{itemize}
\item \alertblue{Dynamic power}. Power dissipated in \acp{IC} due to switching activity.
\begin{itemize}
  \item Charging and discharging load capacitances.
  \item Short-circuit current, when both pMOS and nMOS transistors are on for a brief period.
\end{itemize}
\item \alertblue{Static power}. Power dissipated in \acp{IC} even when there is no switching activity.
\begin{itemize}
  \item Power is consumed even if transistors are off.
  \item Gate and junction leakages.
\end{itemize}
\item \alertblue{Total power}.
\end{itemize}
\begin{equation}
P_{\mathrm{TOTAL}} = P_{\mathrm{dyn}} + P_{\mathrm{static}}
\end{equation}
\end{frame}

% ====================
% Dynamic power
% ====================
\begin{frame}{Dynamic power}{}
\alertblue{Dynamic power}
\begin{itemize}
\item Dynamic power is described in \eref{eq:DynamicPower}
\begin{equation}
\label{eq:DynamicPower}
P_{dyn} = \alpha \cdot C_{L} \cdot {V_{DD}}^{2} \cdot f_{clk}
\end{equation} 
\item[]where
\item[]$\alpha \in [0,1]$ is a switching activity factor respect to a clock source, \ie, clock signals have $\alpha=1$. It determines the probability of a net transitioning from 0 to 1,
\item[]$C_{L}$ is the capacitive load,
\item[]${V_{DD}}$ is the supply voltage,
\item[]$f_{clk}$ is the clock frequency.
\end{itemize}
\end{frame}

% ====================
% DVFS
% ====================
\begin{frame}{Low-power design techniques}{}
\alertblue{\acl{DVFS}}
\begin{figure}
\includegraphics[scale=0.6]{LowPower_Inverter}
\caption{\acs{CMOS} inverter showing a load capacitance.}
\label{Figure:CMOS_inverter}
\end{figure}
\end{frame}

% ====================
% Low-power techniques
% ====================
\begin{frame}{Low-power design techniques}{}
\alertblue{Low-power design techniques}
\begin{itemize}
\item \alertblue{\ac{DVFS}} for reducing dynamic power.
\item \alertblue{Clock gating} for reducing dynamic power .
\item \alertblue{Power gating} for reducing static power.
\end{itemize}
\end{frame}

\section{Dynamic Voltage and Frequency Scaling}
% ====================
% DVFS
% ====================
\begin{frame}{Low-power design techniques}{}
\alertblue{\acl{DVFS}}
\begin{itemize}
\item From \eref{eq:DynamicPower}, we can see that the supply voltage has a quadratic effect on power dissipation.
\begin{equation*}
P_{dyn} = \alpha \cdot C_{L} \cdot {V_{DD}}^{2} \cdot f_{clk}
\end{equation*}
\item[]As a result, we may simply lower $V_{DD}$.
\item Example: What is the power saving in a system that reduces $V_{DD}$ from 1~V to 0.85~V?
\pause
\begin{equation*}
\frac{P_{dyn2}}{P_{dyn1}} = \frac{\alpha \cdot C_{L} \cdot 0.85^{2} \cdot f_{clk}}{\alpha \cdot C_{L} \cdot 1^{2} \cdot f_{clk}} = 0.7225
\end{equation*}
In other words,
\begin{equation*}
P_{dyn2}= 0.7225 \times P_{dyn1} 
\end{equation*}
\end{itemize}
\end{frame}

% ====================
% DVFS
% ====================
\begin{frame}{Low-power design techniques}{}
\alertblue{\acl{DVFS}}
\begin{itemize}
\item In the previous example, we are reducing power consumption by 27.75\% by reducing only 15\% of the power supply.
\item However, we can't simply reduce $V_{DD}$ without consequences.
\pause
\item Propagation delays in \acf{CMOS} transistors are an inversely proportional function of $V_{DD}$.
\begin{itemize}
  \item Larger $V_{DD}$ values yield smaller propagation delays.
\end{itemize}
\item As a result of this, $f_{clk}$ must be adapted.
\end{itemize}
\end{frame}

% ====================
% DVFS
% ====================
\begin{frame}{Low-power design techniques}{}
\alertblue{\acl{DVFS}}
\begin{figure}
\includegraphics[scale=1.5]{LowPower_VDD_delay}
\caption{Power supply and delay relation.}
\label{Figure:VDD_delay}
\end{figure}
\end{frame}

% ====================
% DVFS
% ====================
\begin{frame}{Low-power design techniques}{}
\alertblue{\acl{DVFS}}
\begin{figure}
\includegraphics[scale=1.5]{LowPower_TimingError}
\caption{Timing error due to excessive delay.}
\label{Figure:Timing error}
\end{figure}
\end{frame}

% ====================
% DVFS
% ====================
\begin{frame}{Low-power design techniques}{}
\alertblue{\acl{DVFS}}
\begin{itemize}
\item \ac{DVFS} consists of modifying both $V_{DD}$ and $f_{clk}$ according to the processing and power requirements of the system.
\end{itemize}
\begin{figure}
\includegraphics[scale=0.3]{LowPower_DVFS_block}
\caption{DVFS basic block diagram.}
\label{Figure:DVFS_block}
\end{figure}

\end{frame}

% ====================
% DVFS
% ====================
\begin{frame}{Low-power design techniques}{}
\alertblue{\acl{DVFS}}
\begin{itemize}
\item A power reduction in a \ac{DVFS} system is achieved by reducing both the supply voltage and the clock frequency.
\item Voltage and frequency pairs are usually stored inside the \ac{DVFS} controller.
\item These values may be pre-loaded or may be calculated on-the-fly using performance statistics through performance evaluation algorithms.
\end{itemize}
\end{frame}

% ====================
% DVFS
% ====================
\begin{frame}{Low-power design techniques}{}
\alertblue{\acl{DVFS}}
\begin{itemize}
\item Recalling on \eref{eq:DynamicPower}
\begin{equation*}
P_{dyn} = \alpha \cdot C_{L} \cdot {V_{DD}}^{2} \cdot f_{clk}
\end{equation*}
\item Example: What is the power saving in a system that reduces $V_{DD}$ by 15\% and $f_{clk}$ by 10\%?
\pause
\begin{equation*}
V_{DD2}=0.85 \cdot V_{DD1}
\end{equation*}
\begin{equation*}
f_{clk2}=0.9 \cdot f_{clk1}
\end{equation*}
\pause
\begin{equation*}
\frac{P_{dyn2}}{P_{dyn1}} = \frac{\alpha \cdot C_{L} \cdot {(0.85V_{DD1})}^{2} \cdot 0.9f_{clk1}}{\alpha \cdot C_{L} \cdot {V_{DD1}}^{2} \cdot f_{clk1}} = 0.65025
\end{equation*}
In other words, dynamic power is reduced $\approx 35\%$ 
\end{itemize}
\end{frame}

% ====================
% DVFS
% ====================
\begin{frame}{Low-power design techniques}{}
\alertblue{\acl{DVFS}}
\begin{itemize}
\item Overclocking might be seen as the opposite of reducing the voltage and frequency for energy saving.
\item Overclocking implies increasing the clock speed and operating voltage beyond the recommended manufacturing settings in order to increase the performance of an \ac{IC}.
\item Overclocking will always yield a higher power consumption.

\end{itemize}
\end{frame}

\section{Clock Gating}
% ====================
% Clock gating
% ====================
\begin{frame}{Low-power design techniques}{}
\alertblue{Clock gating}
\begin{itemize}
\item Another factor that can be easily modified in \eref{eq:DynamicPower} is the switching activity $\alpha$.
\begin{equation*}
P_{dyn} = \alpha \cdot C_{L} \cdot {V_{DD}}^{2} \cdot f_{clk}
\end{equation*}
\item This may be achieved at the algorithmic level or at the \ac{RTL} level.
\item From the circuit point of view, we can disable the clock signal in certain blocks within the \ac{IC} that are idle.\\
\item This prevents any switching activity  and prevents and dynamic power dissipation.
\end{itemize}
\end{frame}

% ====================
% Clock gating
% ====================
\begin{frame}{Low-power design techniques}{}
\alertblue{Clock gating}
\begin{itemize}
\item Another factor that can be easily modified in \eref{eq:DynamicPower} is the switching activity $\alpha$.
\begin{equation*}
P_{dyn} = \alpha \cdot C_{L} \cdot {V_{DD}}^{2} \cdot f_{clk}
\end{equation*}
\item This may be achieved at the algorithmic level or at the \ac{RTL} level.
\item From the circuit point of view, we can disable the clock signal in certain blocks within the \ac{IC} that are idle.\\
\item This prevents any switching activity  and prevents and dynamic power dissipation.
\end{itemize}
\end{frame}

% ====================
% Clock gating
% ====================
\begin{frame}{Low-power design techniques}{}
\vspace{-5pt}
\alertblue{Clock gating}
\begin{figure}
\includegraphics[scale=1]{LowPower_CG1}
\end{figure}
\vspace{-15pt}
\begin{figure}
\includegraphics[width=0.9\textwidth]{LowPower_CG1_wave}
\end{figure}
\end{frame}

% ====================
% Clock gating
% ====================
\begin{frame}{Low-power design techniques}{}
\vspace{-5pt}
\alertblue{Clock gating}
\begin{figure}
\includegraphics[scale=1]{LowPower_CG1}
\end{figure}
\vspace{-15pt}
\begin{figure}
\includegraphics[width=0.9\textwidth]{LowPower_CG1_waveb}
\end{figure}
\end{frame}

% ====================
% Clock gating
% ====================
\begin{frame}{Low-power design techniques}{}
\vspace{-5pt}
\alertblue{Clock gating}
\begin{figure}
\includegraphics[scale=1]{LowPower_CG2}
\end{figure}
\vspace{-10pt}
\begin{figure}
\includegraphics[width=0.85\textwidth]{LowPower_CG2_wave}
\end{figure}
\end{frame}

% ====================
% Clock gating
% ====================
\begin{frame}{Low-power design techniques}{}
\vspace{-5pt}
\alertblue{Clock gating}
\begin{figure}
\includegraphics[scale=1]{LowPower_CG3}
\end{figure}
\vspace{-10pt}
\begin{figure}
\includegraphics[width=0.85\textwidth]{LowPower_CG3_wave}
\end{figure}
\end{frame}

% ====================
% Clock gating
% ====================
\begin{frame}{Low-power design techniques}{}
\alertblue{Clock gating}
\begin{itemize}
\item Fortunately, designers don't have to connect latches, \code{AND} and flip-flops together every time they want to employ clock gating.
\item Standard cell libraries provide clock gating cells.
\end{itemize}
\begin{figure}
\includegraphics[scale=1]{LowPower_CG_cell}
\caption{Clock cell gate.}
\label{Figure:CG_cell}
\end{figure}
\begin{itemize}
\item Moreover, modern synthesizers are clever enough to detect and infer clock-gating cells when reading \ac{RTL}. 
\end{itemize}
\end{frame}

% ====================
% Clock gating
% ====================
\begin{frame}[fragile]{Low-power design techniques}{}
\alertblue{Clock gating}
\begin{itemize}
\item Clock gating may be easily implemented using a combination of \ac{RTL} coding style and synthesizer commands.
\end{itemize}
    \lstset{
    numbers=none,
    captionpos=t,
    title=Clock gating in RTL,
    xleftmargin=.2\textwidth, xrightmargin=.2\textwidth
  }
  \begin{lstlisting}
    always_ff @ (posedge clk)
        if(enable)
            Q <= D;
  \end{lstlisting}
  
\begin{itemize}
\item Synthesis tools such as Synopsys Design Compiler use commands such as
\begin{itemize}
  \item[] \code{set\_clock\_gating\_style -global -minimum\_bitwidth <value>}
\end{itemize}
 
\end{itemize}

\end{frame}

\section{Power Gating}
% ====================
% Power gating
% ====================
\begin{frame}{Low-power design techniques}{}
\alertblue{Power gating}
\begin{itemize}
\item Power gating consists on switching off entire parts of the \ac{IC} by temporarily disconnecting it from the supply rail.
\end{itemize}
\begin{figure}
\includegraphics[scale=0.38]{LowPower_PG}
\caption{Power gating basic block diagram.}
\label{Figure:PG_block}
\end{figure}
\end{frame}

% ====================
% Power gating
% ====================
\begin{frame}{Low-power design techniques}{}
\alertblue{Power gating}
\begin{itemize}
\item Header switch  requires careful design.
\begin{itemize}
  \item Low leakage.
  \item Fast switching on times.
\end{itemize}
\item This technique is only efficient when blocks must be turned off for long periods.
\item When a block is turned off, its state must be saved in order to allow resuming execution.
\end{itemize}

\end{frame}

% ====================
% Power gating
% ====================
\begin{frame}{Low-power design techniques}{}
\alertblue{Other techniques}
\begin{itemize}
\item Multiple $Vt$ design.
\begin{itemize}
  \item Different transistors use different threshold voltages ($Vt$).
\end{itemize}
\item Multi-voltage design
\begin{itemize}
  \item Typically, \acp{SoC} use different power supplies on different blocks and peripherals.
  \item \ac{RAM} access must be fast.
  \item \ac{IO} devices are typically slower than internal memory.
  \item Use higher voltages on fast blocks.
  \item Use lower voltages on slower blocks.
\end{itemize}
\end{itemize}
\end{frame}

% ====================
% Power gating
% ====================
\begin{frame}{Low-power design techniques}{}
\alertblue{Summary}
\begin{itemize}
\item Power is a critical design parameter.
\item Dynamic power is mostly influenced by voltage, clock frequency, and switching activity.
\item Low-power design techniques to remember.
\begin{itemize}
  \item \alertblue{\acl{DVFS}}. Adjust both voltage and frequency.
  \item \alertblue{Clock gating}. Disable clock signal on parts of the \ac{IC}.
  \item \alertblue{Power gating}. Temporary disconnect blocks from power supply.
\end{itemize}
\end{itemize}
\end{frame}
%% ====================
%% ISA vs uA
%% ====================
%\begin{frame}{\acl{ISA}}{Same \ac{ISA}, different \uA - 45~nm technology}
%\begin{figure}[!htb]
%  \begin{minipage}{0.5\textwidth}
%    \centering
%    \begin{itemize}
%      \item x86 \ac{ISA}.
%      \item Quad Core.
%      \item 2.6~GHz.
%      \item 125~W.
%    \end{itemize}
%    \includegraphics[width=0.8\linewidth]{ISA_AMD_Phenom_II_X4.jpg}
%    \caption{AMD Phenom X4}
%    \label{Figure:AMD_Phenom_X4a}
%  \end{minipage}%
%  \begin{minipage}{0.50\textwidth}
%    \centering
%    \begin{itemize}
%      \item x86 \ac{ISA}.
%      \item Dual Core.
%      \item 1.6~GHz.
%      \item 2~W.
%    \end{itemize}
%    \vspace{14mm}
%    \includegraphics[width=0.92\linewidth]{ISA_Intel_Atom.jpg}
%    \caption{Intel Atom}
%    \label{Figure:Intel_Atom}
%  \end{minipage}
%\end{figure}
%\end{frame}




%% ====================
%% ISA characteristics: Registers
%% ====================
%\begin{frame}{Basic registers of a computer}{}
%\begin{figure}
%\includegraphics[scale=0.36]{ISA_basic_processor}
%\caption{Basic structure of a \ac{uP}.}
%\label{Figure:basic_registers}
%\end{figure}
%\end{frame}


\end{document}
