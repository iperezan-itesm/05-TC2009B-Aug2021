\documentclass[number=03]{assignment}
\title{Computer Architecture - TE2003B}
\chead{Assignment 02}
\rhead{ARMv6-M \acs{uA}}
%\date{February - June 2020}

\newif\ifanswers
\answerstrue % comment out to hide answers
\newcommand{\ARMuP}{ARMv6-M\xspace}

\newcommand{\deadline}{23:59 hours on Wednesday April 21st 2021}
% Begin document
\begin{document}

\setcounter{chapter}{1}
\chapter*{Assignment 02 \\ \ARMuP \acl{uA} design}
\acresetall
% ======================================
% Objective
% ======================================
\section{Objectives}
To understand the \ac{uA} of a simplified version of the \ARMuP \ac{uP}. 

% ======================================
% Deadline
% ======================================
\section{Deadline}
\alertblue{\deadline}.
% ======================================
% Teamwork policy
% ======================================
\section{Teamwork policy}
This is an individual assignment. 
% % ======================================
% % Pre-requisites
% % ======================================
% \section{Pre-requisites}
% It is assumed that you are familiar with working with \ModelSim and \Quartus. 
% If you require assistance, you can refer to the first assignment tutorial.
% ======================================
% Background
% ======================================
\section{Background}\label{Sec:Background}
Assume we want to design the \ac{uA} of a simplified \ARMuP \ac{uP}.
More specifically, we are only interested in implementing the eight instructions shown in \tref{Table:ARMv6Instructions} 

\begin{table}[htbp]
    \centering
    \caption{Selected \ARMuP instructions.}
    \label{Table:ARMv6Instructions}
    \begin{tabular}{l|l|l}
      \hline
      Name & Syntax & Meaning \\ \hline \hline
      Addition    & \code{adds rd, rn, rm} & \code{rd = rn + rm} \\ \hline
      Subtraction & \code{subs rd, rn, rm} & \code{rd = rn - rm} \\  \hline
      Bitwise AND & \code{ands rd, rn, rm} & \code{rd = rn \& rm} \\  \hline
      Bitwise OR  & \code{orrs rd, rn, rm} & \code{rd = rn | rm} \\  \hline
      Load        & \code{ldr  rt, rn, imm5} & \code{rt = mem[rn + imm5]} \\ \hline
      Store       & \code{str  rt, rn, imm5} & \code{mem[rn + imm5] = rt } \\ \hline
      Conditional branch & \code{b<cond> imm8} & \code{cond ? PC = imm8 : PC + 2}\\ \hline
      Unconditional branch & \code{b imm8} & \code{PC = imm8}\\ \hline
  	\end{tabular}
  \end{table}


So far, we have designed the \ac{uA} of \fref{Figure:ARMv6_RType_LDR_STR} for performing all R-Type instructions, as well as load and store instructions.

 \begin{figure}[!htb]
  \centering
  \includegraphics[width=\linewidth]{ARMv6_design_RType_LDR_STR_FULL}
  \caption{\uA for performing R-Type, load and store instructions.}
  \label{Figure:ARMv6_RType_LDR_STR}
\end{figure}
 

% ======================================
% ISA and uA
% ====================================== 
\newpage
\section{\ac{uA} improvement}\label{Sec:uA_Improvement}
In this assignment, you are required to improve the schematic of \fref{Figure:ARMv6_RType_LDR_STR}. 
More specifically, you must include the necessary \ac{HW} to \fref{Figure:ARMv6_RType_LDR_STR} in order to perform all eight instructions specified in \tref{Table:ARMv6Instructions}.
\ac{ISA}. 

%
\section{Deliverables and Submission instructions}\label{Sec:Deliverables}
Prepare a \code{pdf} file including.
\begin{enumerate}
\item \marking{50} A schematic drawing showing your final \ac{uA} design.
This schematic must be able to perform all instructions specified in \tref{Table:ARMv6Instructions}.
All your signals should be correctly labelled with a relevant name and a bit width, particularly the signals that are derived from the output of the \ac{IM}.
In order to simplify your schematic diagram, you might opt to use a name-connection scheme for control signals, as shown for \code{writeback\_sel} and \code{rtype\_sel} signals of \tref{Table:ARMv6Instructions}.

\item \marking{50} An explanation of how your \ac{uA} works.
This explanation should include:
\begin{enumerate}
  \item How \code{adds} instruction works.
  \item How \code{ldr} instruction works.
  \item How \code{b<cond>} works.
\end{enumerate}
The quality of the explanation is more important than its quantity!
\end{enumerate} 

\alertblue{IMPORTANT!} Since this assignment involves \ac{uA} design, there may be \alertviolet{multiple} correct answers.

Submit your assignment through Canvas \alertred{no later} than \deadline. 
\\
Please send any questions to \href{mailto:isaac.perez.andrade@tec.mx}{isaac.perez.andrade@tec.mx}.
\end{document}
s