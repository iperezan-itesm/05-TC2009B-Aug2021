\documentclass[number=02]{assignment}
\title{Computer Architecture - TE2031}
\date{February - June 2020}


\newif\ifanswers
\answerstrue % comment out to hide answers
\usepackage{multirow}
% Begin document
\begin{document}

\section*{Multiplication example}
Assume the following initial conditions.
\begin{table}[!h]
\centering
\begin{tabular}{rll}
 & \code{Mem[0] = A} & \\
 & \code{Mem[1] = B} & \\
 & \code{Mem[2] = 0} & \\
 
\end{tabular}
\end{table}

A simple multiplication algorithm is as follows.
\begin{table}[!h]
\centering
\begin{tabular}{rl|l}
 & Instruction & Meaning \\
 \hline
 & \code{LWI r1, (0)} & \code{r1 $\leftarrow$ Mem[0] (r1 = A)} \\
 & \code{LWI r2, (1)} & \code{r2 $\leftarrow$ Mem[1] (r2 = B)} \\
 & \code{LWI r3, (2)} & \code{r3 $\leftarrow$ Mem[2] (r3 = 0)} \\
addition: & \code{ADD r3, r3, r1} & \code{r3 $\leftarrow$ r3 + r1} \\
          & \code{SUBI r2, r2, r1} & \code{r2 $\leftarrow$ r2 - 1} \\
          & \code{BNE r2, r0, addition} & B == 0? \\
          & \code{SWI r3, (2)} & \code{Mem[2] $\leftarrow$ r3}\\
end:      & \code{J end} & \code{Finish program}    
\end{tabular}
\end{table}

\end{document}
