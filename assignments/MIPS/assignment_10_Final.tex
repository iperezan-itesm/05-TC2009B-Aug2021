\documentclass[number=03]{assignment}
\title{Computer Architecture - TE2031}
\chead{Final Project}
\rhead{MIPS programs}
%\date{February - June 2020}

\newif\ifanswers
\answerstrue % comment out to hide answers

% TOP
\newcommand{\mipspkgfile}{\colorfilename{mips\_pkg.sv}}
\newcommand{\mipsfile}{\colorfilename{mips.sv}}

\newcommand{\rtypefile}{\colorfilename{rtype.sv}}
\newcommand{\rtypetbfile}{\colorfilename{tb\_rtype.sv}}
\newcommand{\rtypedofile}{\colorfilename{rtype.do}}

\newcommand{\itypefile}{\colorfilename{itype.sv}}
\newcommand{\itypetbfile}{\colorfilename{tb\_itype.sv}}
\newcommand{\itypedofile}{\colorfilename{itype.do}}

% MUX
\newcommand{\muxfile}[1]{\colorfilename{mux\_#1x1.sv}}
%\newcommand{\mux2x1file}[1]{\code{mux\_2x1.sv}}
\newcommand{\muxpkgfile}{\colorfilename{mux\_pkg.sv}}

% RF
\newcommand{\rfpkgfile}{\colorfilename{rf\_pkg.sv}}
\newcommand{\rffile}{\colorfilename{rf.sv}}
\newcommand{\tbrffile}{\colorfilename{tb\_rf.sv}}
\newcommand{\dorffile}{\colorfilename{rf.do}}

% ALU
\newcommand{\alupkgfile}{\colorfilename{alu\_pkg.sv}}
\newcommand{\alufile}{\colorfilename{alu.sv}}
\newcommand{\tbalufile}{\colorfilename{tb\_alu.sv}}

% SIGN EXTENDER
\newcommand{\sgnextfile}{\colorfilename{sgn\_ext.sv}}
\newcommand{\tbsgnextfile}{\colorfilename{tb\_sgn\_ext.sv}}

% CONTROL UNIT
\newcommand{\controlfile}{\colorfilename{control\_unit.sv}}

% DATA MEMORY
\newcommand{\dmfile}{\colorfilename{dm.sv}}

% INSTRUCTION MEMORY
\newcommand{\imfile}{\colorfilename{im.sv}}

% PROGRAM COUNTER
\newcommand{\pcfile}{\colorfilename{program\_counter.sv}}

% TESTBENCHES 
\newcommand{\mipstbfile}{\colorfilename{tb\_mips.sv}}
\newcommand{\multfile}{\colorfilename{tb\_multiplication.sv}}
\newcommand{\fibonaccifile}{\colorfilename{tb\_fibonacci.sv}}
\newcommand{\sumsquaresfile}{\colorfilename{tb\_sum\_squares.sv}}

\newcommand{\mipsbinfile}{\colorfilename{mips\_tests.txt}}
\newcommand{\multbinfile}{\colorfilename{multiplication.txt}}
\newcommand{\fibonaccibinfile}{\colorfilename{fibonacci.txt}}
\newcommand{\sumsquaresbinfile}{\colorfilename{sum\_squares.txt}}

\newcommand{\mipsdofile}{\colorfilename{mips.do}}
\newcommand{\multdofile}{\colorfilename{multiplication.do}}
\newcommand{\fibonaccidofile}{\colorfilename{fibonacci.do}}
\newcommand{\sumsquaresdofile}{\colorfilename{sum\_squares.do}}

% GRADE WEIGHT
\newcommand{\DeliveryWeight}{This delivery constitutes 25\% of your final grade.}

% DEADLINE
\newcommand{\deadline}{23:59 hours on Thursday June 4th 2020}


\newcommand{\Fibonacci}{Fibonacci\xspace}

\makesavenoteenv{tabular}
\makesavenoteenv{table}
% Begin document
\begin{document}

\setcounter{chapter}{1}
\chapter*{Final Project \\ MIPS}
% ======================================
% Objective
% ======================================
\section{Objectives}
\begin{itemize}
\item To implement the datapath of a custom \ac{MIPS} in \SV.  
\item To implement a \ac{MIPS} assembler program in order to multiply two signed numbers.
\item To implement a \ac{MIPS} assembler program in order to calculate the first $n$-th \Fibonacci numbers.
\item To implement a \ac{MIPS} assembler program in order to calculate the first $n$-th element in a sum-of-square series.
\end{itemize}


% ======================================
% Final Partial weight
% ======================================
\section{Final grade weight}
\alertblue{\DeliveryWeight}
% ======================================
% Deadline
% ======================================
\section{Deadline}
\alertblue{\deadline}

% ======================================
% Teamwork policy
% ======================================
\section{Teamwork policy}
This is a group assignment. 
% ======================================
% Pre-requisites
% ======================================
\section{Pre-requisites}
It is assumed that you are familiar with working with \ModelSim and \Quartus. 
If you require assistance, you can refer to the first assignment tutorial.
It is assumed that you have completed the previous assignment for modelling the \ac{MIPS} \Rtype and \Itype instructions in \SV.

%\acresetall
\newpage
% ======================================
% Background
% ======================================
\section{Background}
The following sections provide some background on the tasks you are required to complete for this assignment.

\subsection{Fibonacci sequence}
A \Fibonacci number $F_{n}$ is defined as the sum of the two previous \Fibonacci numbers as defined in \eref{Equation:Fibonacci}

\begin{equation}
\label{Equation:Fibonacci}
F_{n} = F_{n-1} + F_{n-2}
\end{equation}

with $n>=2$ and $F_{0} = 0$ and $F_{1} = 1$.

Moreover, the $n$-th order \Fibonacci sequence is defined as $\{F_{0}, F_{1}, F_{2}, \dots, F_{n-2},  F_{n-1}, F_{n}\}$. 
For example, the $5$-th order \Fibonacci sequence is $\{F_{0}, F_{1}, F_{2}, F_{3},  F_{4}, F_{5}\} = \{0, 1, 1, 2, 3, 5\}$.

\subsection{Sum of squares}
The $n$-th term in a sum of square series is defined as in \eref{Equation:SumSquare}
\begin{equation}
\label{Equation:SumSquare}
S_{n} = \sum_{i=1}^{n}i^{2} = 1^{2} + 2^{2} + 3^{2} + \dots + (n-2)^{2} + (n-1)^{2} + n^{2}
\end{equation}

Moreover, the $n$-th order sum of square sequence is defined as $\{S_{1}, S_{2}, \dots, S_{n-2},  S_{n-1}, S_{n}\}$.
For example, the $5$-th order sum of square sequence is $\{S_{1}, S_{2}, S_{3},  S_{4}, S_{5}\} = \{1, 5, 14, 30, 55\}$.

% ======================================
% Specifications
% ======================================
\section{Project specifications}
The following sections provide an overview of the design specifications for this assignment.

\subsection{General specifications}\label{sec:general_specs}
Your custom \ac{MIPS} design must comply with the following design specifications.
%
\begin{itemize}
\item Instruction set based (but modified) on a standard 16-bit \ac{MIPS} \ac{uP}.
\item Single-cycle design.
\item Data width is 16-bits.
\item Instruction encoding (instruction width) is 32-bits long.
\item \ac{RF} contains 32 registers.
\begin{itemize}
\item \code{r0} \alertblue{must} always be 0, \ie, \code{r0} is not writeable.
\item \code{r31} is return address.
\end{itemize}
\item \ac{IM} contains 256 address.
\item \ac{DM} contains 256 address.
\end{itemize}
%

\newpage
\subsection{MIPS top-level}

\tref{Table:toplevel_ports} specifies the top-level ports required for this assignment.
%
\begin{table}[!htb]
\centering
\caption{Top-level ports.}
\label{Table:toplevel_ports}
\begin{tabular}{l|l|c|l}
\hline\hline
Port name & Direction & Width & Description \\
\hline\hline
\code{clk}          & \code{input}  & 1  & Clock signal \\ \hline
\code{asyn\_n\_rst} & \code{input}  & 1  & Asynchronous active-low reset \\ \hline
 \end{tabular}
\end{table}
%

This top-level module consists of only inputs \clk and \asynnrst.
There are no outputs in this top-level module.
You must include the \ac{IM} and the \ac{PC} as part of your \ac{MIPS} design.

\subsection{List of instructions}\label{sec:Instructions}
\tref{Table:Instructions} shows the minimum set of instructions that your \ac{MIPS} should support.
You may support additional instructions that you deem necessary. 


\begin{table}[!htb]
\centering
\caption{Required \ac{MIPS} instructions.}
\label{Table:Instructions}
\begin{tabular}{l|l|l}
\hline\hline
 Instruction & Syntax & Meaning \\
 \hline\hline
 \multicolumn{3}{c}{\Rtype}\\\hline
 \code{ADD}  & \code{ADD rd, rs, rt}  & \code{Reg[rd] $\leftarrow$ Reg[rs] + Reg[rt]}\\ \hline
    \code{SUB}  & \code{SUB rd, rs, rt}  & \code{Reg[rd] $\leftarrow$ Reg[rs] - Reg[rt]}\\ \hline
    \code{NAND} & \code{NAND rd, rs, rt}  & \code{Reg[rd] $\leftarrow$ $\sim$(Reg[rs] \& Reg[rt])}\\ \hline
    \code{NOR}  & \code{NOR rd, rs, rt}  & \code{Reg[rd] $\leftarrow$ $\sim$(Reg[rs] | Reg[rt])}\\ \hline
    \code{XNOR} & \code{XNOR rd, rs, rt} & \code{Reg[rd] $\leftarrow$ $\sim$(Reg[rs] \^{} Reg[rt])}\\ \hline
    \code{AND}  & \code{AND rd, rs, rt}  & \code{Reg[rd] $\leftarrow$ Reg[rs] \& Reg[rt]}\\ \hline
    \code{OR}   & \code{OR rd, rs, rt}   & \code{Reg[rd] $\leftarrow$ Reg[rs] | Reg[rt]}\\ \hline
    \code{XOR}  & \code{XOR rd, rs, rt}  & \code{Reg[rd] $\leftarrow$ Reg[rs] \^{} Reg[rt]}\\ \hline
    \code{SLL}  & \code{SLL rd, rs, sa} & \code{Reg[rd] $\leftarrow$ Reg[rt] $<<$ sa}\\ \hline
    \code{SRL}  & \code{SLL rd, rs, sa} & \code{Reg[rd] $\leftarrow$ Reg[rt] $>>$ sa}\\ \hline
    \code{SLA}  & \code{SLL rd, rs, sa} & \code{Reg[rd] $\leftarrow$ Reg[rt] $<<<$ sa}\\ \hline
    \code{SRA}  & \code{SLL rd, rs, sa} & \code{Reg[rd] $\leftarrow$ Reg[rt] $>>>$ sa}\\ \hline
    \code{JR} & \code{JR rs}  & \code{PC $\leftarrow$ Reg[rs]} \\
    \hline     
    \hline
    \multicolumn{3}{c}{\Itype}\\\hline
    \code{ADDI} & \code{ADDI rt, rs, imm} & \code{Reg[rt] $\leftarrow$ Reg[rs] + imm}    \\\hline
    \code{SUBI} & \code{SUBI rt, rs, imm} & \code{Reg[rt] $\leftarrow$ Reg[rs] - imm}    \\\hline
    \code{NANDI} & \code{ANDI rt, rs, imm} & \code{Reg[rt] $\leftarrow$ $\sim$(Reg[rs] \& imm)}   \\\hline
    \code{NORI}  & \code{ORI  rt, rs, imm} & \code{Reg[rt] $\leftarrow$ $\sim$(Reg[rs] | imm)}    \\\hline
    \code{XNORI} & \code{XORI rt, rs, imm} & \code{Reg[rt] $\leftarrow$ $\sim$(Reg[rs] \^{} imm)} \\\hline
    \code{ANDI} & \code{ANDI rt, rs, imm} & \code{Reg[rt] $\leftarrow$ Reg[rs] \& imm}   \\\hline
    \code{ORI}  & \code{ORI  rt, rs, imm} & \code{Reg[rt] $\leftarrow$ Reg[rs] | imm}    \\\hline
    \code{XORI} & \code{XORI rt, rs, imm} & \code{Reg[rt] $\leftarrow$ Reg[rs] \^{} imm} \\\hline
    \code{LUI}  & \code{LUI rt, imm}      & \code{Reg[rt] $\leftarrow$ \{imm[7:0], 8'b0\}}   \\\hline
    \code{LLI}  & \code{LLI rt, imm}      & \code{Reg[rt] $\leftarrow$ \{8'b0, imm[7:0]\}}  \\\hline
    \code{LI}  & \code{LI rt, imm}      & \code{Reg[rt] $\leftarrow$ imm}   \\\hline
    \code{LW}   & \code{LW  rt, imm(rs)}  & \code{Reg[rt] $\leftarrow$ Mem[Reg[rs]+imm]} \\\hline
    \code{SW}   & \code{SWR  rt, imm(rs)} & \code{Mem[Reg[rs]+imm] $\leftarrow$ Reg[rt]} \\\hline
    \multirow{2}{*}{\code{BEQ}} & \multirow{2}{*}{\code{BEQ rt, rs, imm}}  & \code{if (Reg[rs] == Reg[rt])}  \\
    & & \code{then PC $\leftarrow$ imm} \\\hline
    \multirow{2}{*}{\code{BNEQ}} & \multirow{2}{*}{\code{BNEQ rt, rs, imm}}  & \code{if (Reg[rs] != Reg[rt])}  \\
    & & \code{then PC $\leftarrow$ imm} \\\hline
    \multirow{2}{*}{\code{BZ}} & \multirow{2}{*}{\code{BZ rt, rs, imm}}  & \code{if (Reg[rs] == 0)}  \\
    & & \code{then PC $\leftarrow$ imm} \\\hline
    \multirow{2}{*}{\code{BNEG}} & \multirow{2}{*}{\code{BNEG rt, rs, imm}}  & \code{if (Reg[rs] < 0)}  \\
    & & \code{then PC $\leftarrow$ imm} \\
    \hline
    \hline
     \multicolumn{3}{c}{\Jtype}\\\hline
    \code{J}   & \code{J imm} & \code{PC $\leftarrow$ imm}\\\hline 
    \multirow{2}{*}{\code{JAL}} & \multirow{2}{*}{\code{JAL imm}}  & \code{Reg[31] $\leftarrow$ PC+1}  \\
    & & \code{PC $\leftarrow$ imm} \\\hline 
 \end{tabular}
\end{table}

\subsection{MIPS test programs}\label{sec:MIPS_programs}
The following sections describe three \ac{MIPS} test programs that you must complete for this assignment.
All three programs should be designed using assembly language with the instructions you have designed in your custom \ac{MIPS}.
You must follow the mnemonic convention for each instruction in \tref{Table:Instructions}. 

\subsubsection{Multiplication}
You are required to design an assembly program for multiplying 2 signed numbers.
For this part of the assignment you must consider that the two operands $A$ and $B$ are placed in \code{Mem[0]} and \code{Mem[1]}. 
As a result of this, your testbench must initialize your \ac{DM} with the two operands. 
The result of the signed multiplication must be stored in \code{Mem[2]}.

\subsubsection{\Fibonacci sequence}
You are required to design an assembly program for calculating the $n$-th order \Fibonacci sequence.
For this part of the assignment, you must consider that the value of $n$ is stored in \code{Mem[255]}. 
As a result of this, your testbench must initialize \code{Mem[255]} with the value of $n$.
The $n+1$ elements of the \Fibonacci sequence must be stored from \code{Mem[0]} to \code{Mem[$n$]}. 
For example, for the $5$-th order \Fibonacci sequence, the results must be stored in \code{Mem[0], Mem[1], Mem[2], Mem[3], Mem[4], Mem[5]}.

\subsubsection{Sum of squares sequence}
You are required to design an assembly program for calculating the $n$-th order sum of squares sequence.
For this part of the assignment, you must consider that the value of $n$ is stored in \code{Mem[0]}. 
As a result of this, your testbench must initialize \code{Mem[0]} with the value of $n$.
The $n$ elements of the sum of squares sequence must be stored from \code{Mem[1]} to \code{Mem[$n$]}. 
For example, for the $5$-th order sum of squares sequence, the results must be stored in \code{Mem[1], Mem[2], Mem[3], Mem[4], Mem[5]}.

\subsection{\SV design and testbench files}\label{sec:SV_files}
The minimum required \SV design files and testbenches are listed below.
You may decide to create new \SV modules and files in order to create different hierarchy levels. 
For example, you may decide to create a \SV module specifically for instruction decoding inside the control unit.
\begin{itemize}
\item Desing units.
\begin{itemize}
  \item \mipspkgfile. Package file common to all design files. You must declare all the necessary parameters and data types in this file.
  \item \mipsfile. Top-level \ac{MIPS} module.
  \item \alufile. \ac{ALU} module description.
  \item \rffile. \ac{RF} module description.
  \item \sgnextfile. Zero/sign extender module description.
  \item \muxfile{2}. 2x1 \ac{MUX} module description.
  \item \muxfile{4}. 4x1 \ac{MUX} module description (if needed).
  \item \muxfile{8}. 8x1 \ac{MUX} module description (if needed).
  \item \controlfile. Control unit module description.
  \item \dmfile. \ac{DM} module description.
  \item \imfile. \ac{IM} module description.
  \item \pcfile. \ac{PC} module description.
  \end{itemize}
\item Testbenches.
  \begin{itemize}
  \item \mipstbfile. Top-level \ac{MIPS} testbench. This testbench tests all possible \ac{MIPS} instructions. 
  \item \multfile. Testbench for testing the multiplication program.
  \item \fibonaccifile. Testbench for testing the \Fibonacci program.
  \item \sumsquaresfile. Testbench for testing the sum of squares program.
  \end{itemize}
\item \ac{IM} initialization files. These files may be in either binary or hexadecimal format.
  \begin{itemize}
  \item \mipsbinfile. File for loading \ac{IM} in \mipstbfile. 
  \item \multbinfile. File for loading \ac{IM} in \multfile. 
  \item \fibonaccibinfile. File for loading \ac{IM} in \fibonaccifile.
  \item \sumsquaresbinfile. File for loading \ac{IM} in \sumsquaresfile.
  \end{itemize}
\end{itemize}

\section{Report}\label{sec:Report}
You are required to submit a report, which must include the following sections.

\begin{enumerate}
  \item \alertblue{Introduction}. An introduction to the report and to the project.
  \item \alertblue{\ac{MIPS} \ac{ISA} design}. An explanation of your \ac{MIPS} \ac{ISA}. This section must include:
  \begin{itemize}
    \item A table showing your instruction encoding.
  This must include \alertred{ALL} your supported instructions.
    \item An explanation of your design choices and design flow.
    \item A computer-drawn schematic diagram showing your final \ac{uA} of your \ac{MIPS} design. 
    \begin{enumerate}
      \item This \ac{uA} must support all \Rtype, \Itype and \Jtype \ac{MIPS} instructions. 
      \item All signals must be clearly labelled, including bus widths.
     \end{enumerate}
  \end{itemize}
  \item \alertblue{Design synthesis.} You must include a screenshot of the generated \ac{RTL} in \Quartus.
  \item \alertblue{Test programs.} For each test program, a table similar to \tref{Table:Testcode} explaining your test files (\code{.txt}) line by line. 
  This explanation must include the binary or hexadecimal value loaded into your \ac{IM} and the test you are providing.
  %
  \begin{table}[!htb]
    \centering
	\caption{Example of test file in your report.}
	\label{Table:Testcode}
	\begin{tabular}{l|l|l}
	  Line & Hex code & Meaning \\
	  \hline\hline
	  1 & \code{FEDCBA98} & \code{ADDI r1, r2, -16} \\\hline
	  2 & \code{76543210} & \code{BNE r3, r4, 123} \\\hline 
	\end{tabular}
  \end{table}
  %
  
  For each test program, you must also state the limitation of your program. For example, you must indicate the maximum value of $n$ that your \ac{MIPS} \ac{uA} supports for the Fibonacci and sum of squares programs. Similarly, you must indicate the maximum and minimum possible results for your multiplication program.
  This section must also include screenshots of your \ModelSim simulation for each test program.
  
  \item \alertblue{Improvements and future work.} Mention any potential improvements to you \ac{MIPS} \ac{ISA} and/or \ac{uA}.
  \item \alertblue{Conclusions.} Concluding remarks on what you've learned in this project.
    \item \alertblue{References.} List of citations in \href{http://journals.ieeeauthorcenter.ieee.org/wp-content/uploads/sites/7/IEEE-Reference-Guide.pdf}{IEEE} format used for the report.
\end{enumerate}

\alertred{NOTE.} There is no restriction on the number of pages. 
However, quality is prioritized over quantity. 
A lengthy but meaningless report will result in a lower score than a shorter but more meaningful report.

% ======================================
% Grading criteria
% ====================================== 
\section{Grading criteria}\label{sec:Grading}
The grade of this project is divided as presented in \tref{Table:Grading}.
  %
  \begin{table}[!htb]
    \centering
	\caption{Grading criteria.}
	\label{Table:Grading}
	\begin{tabular}{l|r}
	\hline
	  Rubric & Percentage \\
	  \hline\hline
	  \Jtype implementation  \& testbench & 5\% \\\hline
	  Multiplication program              & 25\% \\\hline
	  Fibonacci program                   & 25\% \\\hline
	  Sum of squares program              & 25\% \\\hline
	  Report                              & 20\% \\\hline\hline
	  TOTAL                               & 100\% \\  
	\end{tabular}
  \end{table}
  %
 
In addition to the grading criteria specified in  \tref{Table:Grading}, I will consider the following aspects.
\begin{enumerate}
\item \alertblue{The correct functionality of your designs.}
I will use my own testbenches in order to automatically stress your designs and verify that they perform the tasks according to the specifications. 
For example, I will try different values for the parameters in your designs and I expect them to still perform according to the specifications.
This is why it is paramount that you follow the name convention specified for file names and port names.
Moreover, it is important that your designs and testbenches compile in \ModelSim without warnings and without errors.
\alertred{Each \acs{RTL} warning will deduct 5\% of your final project grade.}
\alertred{Your maximum grade for this assignment will automatically drop to 50/100 should \ModelSim trigger a compilation or simulation error.}
\item \alertblue{The quality of your testbenches.}
Even though I will use my own testbenches, I expect you to consider a thorough and concious set of test scenarios.
In this way, you should be able to spot any mismatches between the expected results and the actual results provided by your designs.
\item \alertblue{Your designs must be synthesized in \Quartus without latches, without \acs{RTL} warnings and without errors.}
\alertred{Each \acs{RTL} warning will deduct 5\% of your final project grade.}
\alertred{Your maximum grade for this assignment will automatically drop to 50/100 should \Quartus trigger a synthesis error or generate unwanted latches.}
\begin{sloppypar}
The only warnings that will be tolerated are those that are not related to the \acs{RTL} itself. 
Examples of such warnings are:
\begin{itemize}
\item \code{Warning (18236): Number of processors has not been specified which may cause overloading on shared machines. Set the global assignment} \linebreak\code{NUM\_PARALLEL\_PROCESSORS in your QSF to an appropriate value for best performance.}
\item \code{Warning (13046): Tri-state node(s) do not directly drive top-level pin(s).}
\item \code{Warning (292013): Feature LogicLock is only available with a valid subscription license. You can purchase a software subscription to gain full access to this feature.}
\item \code{Warning (15714): Some pins have incomplete I/O assignments. Refer to the I/O Assignment Warnings report for details.}
\end{itemize}
\end{sloppypar}
Remember that a design is not useful if it can't be synthesized.
\end{enumerate}


% ======================================
% Deliverables and Submission instructions
% ====================================== 
\section{Deliverables and Submission instructions}\label{sec:Deliverables}
Prepare a single \code{zip} file containing all your design, testbench and test program files. 
You may follow the list provided in \sref{sec:SV_files} as a reference.
You may have created additional \SV modules as stated in \sref{sec:SV_files}.
If that's the case, please include them in your \code{zip} file.
In addition to the design, testbench and test program files, you must include the following \ModelSim scripts.
\begin{enumerate}
\item \mipsdofile. \ModelSim script for compiling and simulating your design. This will be used for testing the \Jtype instructions, along with the \Rtype and \Itype instructions.
\item \multdofile. \ModelSim script for compiling and simulating your design using the multiplication test program.
\item \fibonaccidofile. \ModelSim script for compiling and simulating your design using the Fibonacci test program.
\item \sumsquaresdofile. \ModelSim script for compiling and simulating your design using the sum of squares test program.
\end{enumerate}

Submit your assignment through Blackboard \alertred{no later} than \deadline.
Only one submission per team is necessary.
\\
Please send any questions to \href{mailto:isaac.perez.andrade@tec.mx}{isaac.perez.andrade@tec.mx}.
\end{document}
