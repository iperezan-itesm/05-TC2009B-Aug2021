\documentclass[number=03]{assignment}
\title{Computer Architecture - TE2031}
\chead{Second Partial Project 3/3}
\rhead{Complete MIPS}
%\date{February - June 2020}

\newif\ifanswers
\answerstrue % comment out to hide answers

% TOP
\newcommand{\mipspkgfile}{\colorfilename{mips\_pkg.sv}}
\newcommand{\mipsfile}{\colorfilename{mips.sv}}

\newcommand{\rtypefile}{\colorfilename{rtype.sv}}
\newcommand{\rtypetbfile}{\colorfilename{tb\_rtype.sv}}
\newcommand{\rtypedofile}{\colorfilename{rtype.do}}

\newcommand{\itypefile}{\colorfilename{itype.sv}}
\newcommand{\itypetbfile}{\colorfilename{tb\_itype.sv}}
\newcommand{\itypedofile}{\colorfilename{itype.do}}

% MUX
\newcommand{\muxfile}[1]{\colorfilename{mux\_#1x1.sv}}
%\newcommand{\mux2x1file}[1]{\code{mux\_2x1.sv}}
\newcommand{\muxpkgfile}{\colorfilename{mux\_pkg.sv}}

% RF
\newcommand{\rfpkgfile}{\colorfilename{rf\_pkg.sv}}
\newcommand{\rffile}{\colorfilename{rf.sv}}
\newcommand{\tbrffile}{\colorfilename{tb\_rf.sv}}
\newcommand{\dorffile}{\colorfilename{rf.do}}

% ALU
\newcommand{\alupkgfile}{\colorfilename{alu\_pkg.sv}}
\newcommand{\alufile}{\colorfilename{alu.sv}}
\newcommand{\tbalufile}{\colorfilename{tb\_alu.sv}}

% SIGN EXTENDER
\newcommand{\sgnextfile}{\colorfilename{sgn\_ext.sv}}
\newcommand{\tbsgnextfile}{\colorfilename{tb\_sgn\_ext.sv}}

% CONTROL UNIT
\newcommand{\controlfile}{\colorfilename{control\_unit.sv}}

% DATA MEMORY
\newcommand{\dmfile}{\colorfilename{dm.sv}}

% INSTRUCTION MEMORY
\newcommand{\imfile}{\colorfilename{im.sv}}

% PROGRAM COUNTER
\newcommand{\pcfile}{\colorfilename{program\_counter.sv}}

% TESTBENCHES 
\newcommand{\mipstbfile}{\colorfilename{tb\_mips.sv}}
\newcommand{\multfile}{\colorfilename{tb\_multiplication.sv}}
\newcommand{\fibonaccifile}{\colorfilename{tb\_fibonacci.sv}}
\newcommand{\sumsquaresfile}{\colorfilename{tb\_sum\_squares.sv}}

\newcommand{\mipsbinfile}{\colorfilename{mips\_tests.txt}}
\newcommand{\multbinfile}{\colorfilename{multiplication.txt}}
\newcommand{\fibonaccibinfile}{\colorfilename{fibonacci.txt}}
\newcommand{\sumsquaresbinfile}{\colorfilename{sum\_squares.txt}}

\newcommand{\mipsdofile}{\colorfilename{mips.do}}
\newcommand{\multdofile}{\colorfilename{multiplication.do}}
\newcommand{\fibonaccidofile}{\colorfilename{fibonacci.do}}
\newcommand{\sumsquaresdofile}{\colorfilename{sum\_squares.do}}

% GRADE WEIGHT
\newcommand{\DeliveryWeight}{This delivery constitutes 5\% of your second partial final grade.}

% DEADLINE
\newcommand{\deadline}{23:59 hours on Wednesday October 28th 2020}


\newcommand{\Fibonacci}{Fibonacci\xspace}

\makesavenoteenv{tabular}
\makesavenoteenv{table}
% Begin document
\begin{document}

\setcounter{chapter}{1}
\chapter*{Second Partial Project \\ Delivery 3/3 \\ Complete MIPS datapath}
% ======================================
% Objective
% ======================================
\section{Objectives}
\begin{itemize}
\item To implement the datapath of a custom \ac{MIPS} in \SV.
\end{itemize}


% ======================================
% Final Partial weight
% ======================================
\section{Second partial grade weight}
\alertblue{\DeliveryWeight}
% ======================================
% Deadline
% ======================================
\section{Deadline}
\alertblue{\deadline}

% ======================================
% Pre-requisites
% ======================================
\section{Pre-requisites}
It is assumed that you are familiar with working with \ModelSim and \Quartus. 
If you require assistance, you can refer to the first assignment tutorial.
It is assumed that you have completed the previous assignment for modelling the \ac{MIPS} \Rtype and \Itype instructions in \SV.

%\acresetall
\newpage
% ======================================
% Specifications
% ======================================
\section{Specifications}
The following sections provide an overview of the design specifications for this assignment.

\subsection{General specifications}\label{sec:general_specs}
Your custom \ac{MIPS} design must comply with the following design specifications.
%
\begin{itemize}
\item Instruction set based (but modified) on a standard 16-bit \ac{MIPS} \ac{uP}.
\item Single-cycle design.
\item Data width is 16-bits.
\item Instruction encoding (instruction width) is 32-bits long.
\item \ac{RF} contains 32 registers.
\begin{itemize}
\item \code{r0} \alertblue{must} always be 0, \ie, \code{r0} is not writeable.
\item \code{r31} is return address.
\end{itemize}
\item \ac{IM} contains 256 address.
\item \ac{DM} contains 256 address.
\end{itemize}
%

\subsection{MIPS top-level}

\tref{Table:toplevel_ports} specifies the top-level ports required for this assignment.
%
\begin{table}[!htb]
\centering
\caption{Top-level ports.}
\label{Table:toplevel_ports}
\begin{tabular}{l|l|c|l}
\hline\hline
Port name & Direction & Width & Description \\
\hline\hline
\code{clk}          & \code{input}  & 1  & Clock signal \\ \hline
\code{asyn\_n\_rst} & \code{input}  & 1  & Asynchronous active-low reset \\ \hline
 \end{tabular}
\end{table}
%

This top-level module consists of only inputs \clk and \asynnrst and should be named \codeblue{mips}.
There are no outputs in this top-level module.
You must include the \ac{IM} and the \ac{PC} as part of your \ac{MIPS} design.

\subsection{List of instructions}\label{sec:Instructions}
\tref{Table:Instructions} shows the minimum set of instructions that your \ac{MIPS} should support.
You may support additional instructions that you deem necessary. 

\begin{table}[!htb]
\centering
\caption{Required \ac{MIPS} instructions.}
\label{Table:Instructions}
\begin{tabular}{l|l|l}
\hline\hline
 Instruction & Syntax & Meaning \\
 \hline\hline
 \multicolumn{3}{c}{\Rtype}\\\hline
 \code{ADD}  & \code{ADD rd, rs, rt}  & \code{Reg[rd] $\leftarrow$ Reg[rs] + Reg[rt]}\\ \hline
    \code{SUB}  & \code{SUB rd, rs, rt}  & \code{Reg[rd] $\leftarrow$ Reg[rs] - Reg[rt]}\\ \hline
    \code{NAND} & \code{NAND rd, rs, rt}  & \code{Reg[rd] $\leftarrow$ $\sim$(Reg[rs] \& Reg[rt])}\\ \hline
    \code{NOR}  & \code{NOR rd, rs, rt}  & \code{Reg[rd] $\leftarrow$ $\sim$(Reg[rs] | Reg[rt])}\\ \hline
    \code{XNOR} & \code{XNOR rd, rs, rt} & \code{Reg[rd] $\leftarrow$ $\sim$(Reg[rs] \^{} Reg[rt])}\\ \hline
    \code{AND}  & \code{AND rd, rs, rt}  & \code{Reg[rd] $\leftarrow$ Reg[rs] \& Reg[rt]}\\ \hline
    \code{OR}   & \code{OR rd, rs, rt}   & \code{Reg[rd] $\leftarrow$ Reg[rs] | Reg[rt]}\\ \hline
    \code{XOR}  & \code{XOR rd, rs, rt}  & \code{Reg[rd] $\leftarrow$ Reg[rs] \^{} Reg[rt]}\\ \hline
    \code{SLL}  & \code{SLL rd, rs, sa} & \code{Reg[rd] $\leftarrow$ Reg[rt] $<<$ sa}\\ \hline
    \code{SRL}  & \code{SLL rd, rs, sa} & \code{Reg[rd] $\leftarrow$ Reg[rt] $>>$ sa}\\ \hline
    \code{SLA}  & \code{SLL rd, rs, sa} & \code{Reg[rd] $\leftarrow$ Reg[rt] $<<<$ sa}\\ \hline
    \code{SRA}  & \code{SLL rd, rs, sa} & \code{Reg[rd] $\leftarrow$ Reg[rt] $>>>$ sa}\\ \hline
    \codeblue{JR} & \codeblue{JR rs}  & \codeblue{PC $\leftarrow$ Reg[rs]} \\
    \hline     
    \hline
    \multicolumn{3}{c}{\Itype}\\\hline
    \code{ADDI} & \code{ADDI rt, rs, imm} & \code{Reg[rt] $\leftarrow$ Reg[rs] + imm}    \\\hline
    \code{SUBI} & \code{SUBI rt, rs, imm} & \code{Reg[rt] $\leftarrow$ Reg[rs] - imm}    \\\hline
    \code{NANDI} & \code{ANDI rt, rs, imm} & \code{Reg[rt] $\leftarrow$ $\sim$(Reg[rs] \& imm)}   \\\hline
    \code{NORI}  & \code{ORI  rt, rs, imm} & \code{Reg[rt] $\leftarrow$ $\sim$(Reg[rs] | imm)}    \\\hline
    \code{XNORI} & \code{XORI rt, rs, imm} & \code{Reg[rt] $\leftarrow$ $\sim$(Reg[rs] \^{} imm)} \\\hline
    \code{ANDI} & \code{ANDI rt, rs, imm} & \code{Reg[rt] $\leftarrow$ Reg[rs] \& imm}   \\\hline
    \code{ORI}  & \code{ORI  rt, rs, imm} & \code{Reg[rt] $\leftarrow$ Reg[rs] | imm}    \\\hline
    \code{XORI} & \code{XORI rt, rs, imm} & \code{Reg[rt] $\leftarrow$ Reg[rs] \^{} imm} \\\hline
    \code{LUI}  & \code{LUI rt, imm}      & \code{Reg[rt] $\leftarrow$ \{imm[7:0], 8'b0\}}   \\\hline
    \code{LLI}  & \code{LLI rt, imm}      & \code{Reg[rt] $\leftarrow$ \{8'b0, imm[7:0]\}}  \\\hline
    \code{LI}  & \code{LI rt, imm}      & \code{Reg[rt] $\leftarrow$ imm}   \\\hline
    \code{LW}   & \code{LW  rt, imm(rs)}  & \code{Reg[rt] $\leftarrow$ Mem[Reg[rs]+imm]} \\\hline
    \code{SW}   & \code{SWR  rt, imm(rs)} & \code{Mem[Reg[rs]+imm] $\leftarrow$ Reg[rt]} \\\hline
    \multirow{2}{*}{\code{BEQ}} & \multirow{2}{*}{\code{BEQ rt, rs, imm}}  & \code{if (Reg[rs] == Reg[rt])}  \\
    & & \code{then PC $\leftarrow$ imm} \\\hline
    \multirow{2}{*}{\code{BNE}} & \multirow{2}{*}{\code{BNEQ rt, rs, imm}}  & \code{if (Reg[rs] != Reg[rt])}  \\
    & & \code{then PC $\leftarrow$ imm} \\\hline
    \multirow{2}{*}{\code{BLEZ}} & \multirow{2}{*}{\code{BLEZ rs, imm}}  & \code{if (Reg[rs] <= 0)}  \\
    & & \code{then PC $\leftarrow$ imm} \\\hline
    \multirow{2}{*}{\code{BGTZ}} & \multirow{2}{*}{\code{BGTZ rs, imm}}  & \code{if (Reg[rs] > 0)}  \\
    & & \code{then PC $\leftarrow$ imm} \\
    \hline
    \hline
     \multicolumn{3}{c}{\Jtype}\\\hline
    \codeblue{J}   & \codeblue{J imm} & \codeblue{PC $\leftarrow$ imm}\\\hline 
    \multirow{2}{*}{\codeblue{JAL}} & \multirow{2}{*}{\codeblue{JAL imm}}  & \codeblue{Reg[31] $\leftarrow$ PC+1}  \\
    & & \codeblue{PC $\leftarrow$ imm} \\\hline 
 \end{tabular}
\end{table}

Note that instructions highlighted in \codeblue{blue} correspond to the new instructions to be added to your existing \code{itype} design.

\newpage
\subsection{\SV design and testbench files}\label{sec:SV_files}
The minimum required \SV design files and testbenches are listed below.
You may decide to create new \SV modules and files in order to create different hierarchy levels. 
For example, you may decide to create a \SV module specifically for instruction decoding inside the control unit.
\begin{itemize}
\item Desing units.
\begin{itemize}
  \item \mipspkgfile. Package file common to all design files. You must declare all the necessary parameters and data types in this file.
  \item \mipsfile. Top-level \ac{MIPS} module.
  \item \alufile. \ac{ALU} module description.
  \item \rffile. \ac{RF} module description.
  \item \sgnextfile. Zero/sign extender module description.
  \item \muxfile{2}. 2x1 \ac{MUX} module description.
  \item \muxfile{4}. 4x1 \ac{MUX} module description (if needed).
  \item \muxfile{8}. 8x1 \ac{MUX} module description (if needed).
  \item \controlfile. Control unit module description.
  \item \dmfile. \ac{DM} module description.
  \item \imfile. \ac{IM} module description.
  \item \pcfile. \ac{PC} module description.
  \end{itemize}
\item Testbenches.
  \begin{itemize}
  \item \mipstbfile. Top-level \ac{MIPS} testbench. This testbench tests all possible \ac{MIPS} instructions. 
  \end{itemize}
\item \ac{IM} initialization file. This files may be in either binary or hexadecimal format.
  \begin{itemize}
  \item \mipsbinfile. File for initializing \ac{IM} in \mipstbfile. 
  \end{itemize}
\end{itemize}

% ======================================
% Grading criteria
% ====================================== 
\section{Grading criteria}\label{sec:Grading}
The following grading criteria will be considered.

\begin{enumerate}
\item \alertblue{The correct functionality of your designs.}
I will use my own testbenches in order to automatically stress your designs and verify that they perform the tasks according to the specifications. 
For example, I will try different values for the parameters in your designs and I expect them to still perform according to the specifications.
This is why it is paramount that you follow the name convention specified for file names and port names.
Moreover, it is important that your designs and testbenches compile in \ModelSim without warnings and without errors.
\begin{enumerate}
\item \alertred{Each \acs{RTL} warning will deduct 5\% of your final project grade.}
\item \alertred{Your maximum grade for this assignment will automatically drop to 50/100 should \ModelSim trigger a compilation or simulation error.}
\end{enumerate}

\item \alertblue{The quality of your testbenches.}
Even though I will use my own testbenches, I expect you to consider a thorough and concious set of test scenarios.
In this way, you should be able to spot any mismatches between the expected results and the actual results provided by your designs.
\item \alertblue{Your designs must be synthesized in \Quartus without latches, without \acs{RTL} warnings and without errors.}
\begin{enumerate}
\item \alertred{Each \acs{RTL} warning will deduct 5\% of your final project grade.}
\item \alertred{Your maximum grade for this assignment will automatically drop to 50/100 should \Quartus trigger a synthesis error or generate unwanted latches.}
\end{enumerate}

\begin{sloppypar}
The only warnings that will be tolerated are those that are not related to the \acs{RTL} itself. 
Examples of such warnings are:
\begin{itemize}
\item \code{Warning (18236): Number of processors has not been specified which may cause overloading on shared machines. Set the global assignment} \linebreak\code{NUM\_PARALLEL\_PROCESSORS in your QSF to an appropriate value for best performance.}
\item \code{Warning (13046): Tri-state node(s) do not directly drive top-level pin(s).}
\item \code{Warning (292013): Feature LogicLock is only available with a valid subscription license. You can purchase a software subscription to gain full access to this feature.}
\item \code{Warning (15714): Some pins have incomplete I/O assignments. Refer to the I/O Assignment Warnings report for details.}
\end{itemize}
\end{sloppypar}
Remember that a design is not useful if it can't be synthesized.
\end{enumerate}


% ======================================
% Deliverables and Submission instructions
% ====================================== 
\section{Deliverables and Submission instructions}\label{sec:Deliverables}
Prepare a single \code{zip} file containing the following items.
\begin{enumerate}
\item \alertred{All your design, testbench and memory initialization files.} 
You may follow the list provided in \sref{sec:SV_files} as a reference.
You may have created additional \SV modules as stated in \sref{sec:SV_files}.
If that's the case, please include them in your \code{zip} file.
\item \mipsdofile. A \ModelSim script for compiling and simulating your design. This will be used for testing the \Jtype instructions, along with the \Rtype and \Itype instructions. You may use several \code{.do} files for different tasks, for example, one \code{.do} for creating a library, one for compiling, one for simulating, one for adding waveforms and one for running all tasks at once.
  \item A computer-drawn schematic diagram showing your final \ac{uA} of your \ac{MIPS} design. 
    \begin{enumerate}
      \item This \ac{uA} must support all \Rtype, \Itype and \Jtype \ac{MIPS} instructions. 
      \item All signals must be clearly labelled, including bus widths.
     \end{enumerate}
  \item Design synthesis. A screenshot of the \Quartus-generated \ac{RTL} view.
\end{enumerate}


Submit your assignment through Canvas \alertred{no later} than \deadline.
Only one submission per team is necessary.
\\
Please send any questions to \href{mailto:isaac.perez.andrade@tec.mx}{isaac.perez.andrade@tec.mx}.
\end{document}
