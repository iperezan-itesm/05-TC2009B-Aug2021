\documentclass[number=1]{exam}
\title{Computer Architecture - TE2031}
\date{February - June 2020}


\newif\ifanswers
\answerstrue % comment out to hide answers



% Begin document
\begin{document}

\section*{TE2031 - \examtitle}
List of commonly used acronyms.
\begin{acronym}
\acro{CISC}{Complex Instruction Set Computer}
\acro{ISA}{Instruction Set Architecture}
\acro{RAM}{Random Access Memory}
\acro{RISC}{Reduced Instruction Set Computer}
\acro{uA}[$\mu$A]{Microarchitecture}
\acro{uP}[$\mu$P]{Microprocessor}
\end{acronym}

\begin{enumerate}
% ========================================

\item \marking{1} How many bits would you require in order to address a 4096 address \acs{RAM}?
\answer{$\ceil{\log_{2}(4096)} = 12$}
  
% ========================================
\item \marking{1} What is \acs{ISA}?
\begin{enumerate}[label=\alph*)]
\item A colour.
\item A heavy metal band.
\item A type of diet.
\multiplechoiseanswer{\item Other. Explain your answer.}
\end{enumerate}
\answer{\acs{ISA} is the link between an application and the physical layer of the computer.} 

% ========================================
\item \marking{1} Describe the relation between \acs{uA} and \acs{ISA}.
\answer{\acs{uA} is defines how the \acs{ISA} is physically implemented.}

% ========================================
\item \marking{1} What is throughput?
\answer{The amount of tasks that we can perform in a given time. More specifically, it is the amount of bits we can output (process) in a given time. It is measured in bits/second.}

% ========================================
\item \marking{1} What is latency?
\answer{The amount of time taken for completing a task. It is measured in seconds.}

% ========================================
\item \marking{1} What is synthesis?
\begin{enumerate}[label=\alph*)]
\item A TV show.
\item A country.
\item A political party.
\multiplechoiseanswer{\item Other. Explain your answer.}
\end{enumerate}
\answer{Synthesis is the process of generating logic gates from a \acs{HDL}.} 

% ========================================
\item \marking{1} What is abstraction?
\begin{enumerate}[label=\alph*)]
\item A video game.
\item A movie.
\item A bestseller author.
\multiplechoiseanswer{\item Other. Explain your answer.}
\end{enumerate}
\answer{Different representations of the same concept in order to deal with different levels of complexity that suits each designer's needs.} 

% ========================================
\item \marking{1} One \acs{ISA} may be implemented using different \uA s.
\begin{enumerate}[label=\alph*)]
\multiplechoiseanswer{\item True.}
\item False.
\end{enumerate}

% ========================================
\item \marking{1} These are some \acs{ISA} characteristics.
\begin{enumerate}[label=\alph*)]
\multiplechoiseanswer{\item Type and size of instructions and operands, instruction encoding, addressing modes, register types.}
\item Instruction encoding, type and size of memory, \acs{uA} implementation, power consumption.
\item Addressing modes, instruction encoding, \acs{uA} implementation, latency.
\item \acs{uA} implementation, addressing modes, type and size of instructions and operands, instruction encoding,  latency.
\end{enumerate}

% ========================================
\item \marking{1} Which of the following sentences is correct.
\begin{enumerate}[label=\alph*)]
\multiplechoiseanswer{\item Instruction encoding defines the \acs{uA} of the \acs{ISA}.}
\item \acs{uA} determines the instruction encoding of an \acs{ISA}.
\end{enumerate}

% ========================================
\item \marking{1} Which types of memory addressing might be used in iterative constructs such as \code{for} loops?
\answer{Autoincrement and autodecrement.}

% ========================================
\item \marking{1} A single instruction encoding may be implemented using different \uA s.
\begin{enumerate}[label=\alph*)]
\item False.
\multiplechoiseanswer{\item True.}
\end{enumerate}

% ========================================
\item \marking{1} By a thorough inspection to an \acs{ISA}, we may be able to extract the following characteristics of a \acs{uP}.
\begin{enumerate}[label=\alph*)]
\item Whether the \acs{uP} is a \acs{CISC} or a \acs{RISC}, chip area, number of registers, types of memory addressing.
\item Types of memory addressing, average number of clock cycles per instruction, size of main memory, power consumption.
\multiplechoiseanswer{\item Memory usage of compiled programs, types of addressing modes.}
\end{enumerate}

% ========================================
\item \marking{1} Define instruction encoding.
\answer{A convention, represented by binary codes, used to distinguish between the different instructions, operands and addressing modes of an \acs{ISA}.}

% ========================================
\item \marking{1} What is the main difference between Harvard a Von-Neumann \acsp{uA}?
\answer{A Harvard \acs{uA} uses separate buses for data and instructions, whilst a Von-Neumann \acs{uA} uses a single bus for both data and instruction.}

% ========================================
\item \marking{1} A \acs{uP} designer has made the decision to increase the number of supported instructions from 32 to 36. What does this design decision implies?
\begin{enumerate}[label=\alph*)]
\item The instruction encoding must be modified.\label{item:q_designimplications_correct1}
\item The technology (size of the transistors) of the \acs{uP} must be modified.\label{item:q_designimplications_incorrect1}
\item The size of the instruction memory must be modified.\label{item:q_designimplications_incorrect2}
\item The \acs{uA} must be modified.\label{item:q_designimplications_correct2}
\item All of the above.\label{item:q_designimplications_incorrect3}
\item None of the above.\label{item:q_designimplications_incorrect4}
\item Options \ref{item:q_designimplications_correct1}, \ref{item:q_designimplications_correct2} and \ref{item:q_designimplications_incorrect2}.\label{item:q_designimplications_incorrect5}
\multiplechoiseanswer{\item Options \ref{item:q_designimplications_correct1} and \ref{item:q_designimplications_correct2}.}
\item Options \ref{item:q_designimplications_correct2}, \ref{item:q_designimplications_incorrect1} and \ref{item:q_designimplications_incorrect2}.\label{item:q_designimplications_incorrect6}


\end{enumerate}

\end{enumerate}
\end{document}
