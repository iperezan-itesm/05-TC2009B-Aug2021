
% Macro for instering in-line code
\def\code#1{\texttt{#1}}
\def\codekw#1{\textcolor{blue}{\texttt{#1}}}
\def\acbf#1{\alertblue{\ac{#1}}}

%\renewcommand{\url}[1]{\textcolor{blue}{\url{#1}}}
\newcommand{\urlblue}[1]{\textcolor{blue}{\url{#1}}}

\newcommand{\VDD}{\ensuremath{\mathrm{V_{DD}}}\xspace}
\newcommand{\VDDX}[1]{\ensuremath{\mathrm{V_{DD#1}}}\xspace}
\newcommand{\Vsupply}[1]{\VDD = #1~V\xspace}
\newcommand{\Tclk}{\ensuremath{\mathrm{T_{clk}}}\xspace}
\newcommand{\TclkX}[1]{\ensuremath{\mathrm{T_{clk#1}}}\xspace}
\newcommand{\Tclkns}[1]{\Tclk = #1~ns\xspace}
\newcommand{\fclk}{\ensuremath{f_\mathrm{{clk}}}\xspace}
\newcommand{\fclkX}[1]{\ensuremath{f_\mathrm{{clk#1}}}\xspace}

% For file names required in assignments
\newcommand{\colorfilename}[1]{\code{\textcolor{purple}{#1}}\xspace}

\newcommand{\module}{\code{\textcolor{blue}{module}}\xspace}
\newcommand{\parameter}{\code{\textcolor{blue}{parameter}}\xspace}
\newcommand{\logic}{\code{\textcolor{blue}{logic}}\xspace}
\newcommand{\typedef}{\code{\textcolor{blue}{typedef}}\xspace}
\newcommand{\package}{\code{\textcolor{blue}{package}}\xspace}
\newcommand{\enum}{\code{\textcolor{blue}{enum}}\xspace}
\newcommand{\assign}{\code{\textcolor{blue}{assign}}\xspace}

\newcommand{\codeblack}[1]{\code{\textcolor{black}{#1}}\xspace}
\newcommand{\codeblue}[1]{\code{\textcolor{blue}{#1}}\xspace}
\newcommand{\codebrown}[1]{\code{\textcolor{brown}{#1}}\xspace}
\newcommand{\codecyan}[1]{\code{\textcolor{cyan}{#1}}\xspace}
\newcommand{\codedarkgray}[1]{\code{\textcolor{darkgray}{#1}}\xspace}
\newcommand{\codegray}[1]{\code{\textcolor{gray}{#1}}\xspace}
\newcommand{\codegreen}[1]{\code{\textcolor{green}{#1}}\xspace}
\newcommand{\codedarkgreen}[1]{\code{\textcolor{green!50!black}{#1}}\xspace}
\newcommand{\codelightgray}[1]{\code{\textcolor{lightgray}{#1}}\xspace}
\newcommand{\codelime}[1]{\code{\textcolor{lime}{#1}}\xspace}
\newcommand{\codemagenta}[1]{\code{\textcolor{magenta}{#1}}\xspace}
\newcommand{\codeolive}[1]{\code{\textcolor{olive}{#1}}\xspace}
\newcommand{\codeorange}[1]{\code{\textcolor{orange}{#1}}\xspace}
\newcommand{\codepink}[1]{\code{\textcolor{pink}{#1}}\xspace}
\newcommand{\codepurple}[1]{\code{\textcolor{purple}{#1}}\xspace}
\newcommand{\codered}[1]{\code{\textcolor{red}{#1}}\xspace}
\newcommand{\codeteal}[1]{\code{\textcolor{teal}{#1}}\xspace}
\newcommand{\codedarkteal}[1]{\code{\textcolor{teal!99!black}{#1}}\xspace}
\newcommand{\codeviolet}[1]{\code{\textcolor{violet}{#1}}\xspace}
\newcommand{\codewhite}[1]{\code{\textcolor{white}{#1}}\xspace}
\newcommand{\codeyellow}[1]{\code{\textcolor{yellow}{#1}}\xspace}

\newcommand{\ModelSim}{ModelSim\xspace}
\newcommand{\Quartus}{Quartus\xspace}

% CPU performance math
\newcommand{\CPImath}{\ensuremath{\mathrm{CPI}}\xspace}
\newcommand{\Itime}{\ensuremath{\mathrm{I}_{t}}\xspace}
% Macro for representing hex numbers
\newcommand{\hex}[1]{\code{#1h}\xspace}

% Colours
\newcommand{\alertblack}[1]{\textbf{\textcolor{black}{#1}}}
\newcommand{\alertblue}[1]{\textbf{\textcolor{blue}{#1}}}
\newcommand{\alertbrown}[1]{\textbf{\textcolor{brown}{#1}}}
\newcommand{\alertcyan}[1]{\textbf{\textcolor{cyan}{#1}}}
\newcommand{\alertdarkgray}[1]{\textbf{\textcolor{darkgray}{#1}}}
\newcommand{\alertgray}[1]{\textbf{\textcolor{gray}{#1}}}
\newcommand{\alertgreen}[1]{\textbf{\textcolor{green}{#1}}}
\newcommand{\alertdarkgreen}[1]{\textbf{\textcolor{green!50!black}{#1}}}
\newcommand{\alertlightgray}[1]{\textbf{\textcolor{lightgray}{#1}}}
\newcommand{\alertlime}[1]{\textbf{\textcolor{lime}{#1}}}
\newcommand{\alertmagenta}[1]{\textbf{\textcolor{magenta}{#1}}}
\newcommand{\alertolive}[1]{\textbf{\textcolor{olive}{#1}}}
\newcommand{\alertorange}[1]{\textbf{\textcolor{orange}{#1}}}
\newcommand{\alertpink}[1]{\textbf{\textcolor{pink}{#1}}}
\newcommand{\alertpurple}[1]{\textbf{\textcolor{purple}{#1}}}
\newcommand{\alertred}[1]{\textbf{\textcolor{red}{#1}}}
\newcommand{\alertteal}[1]{\textbf{\textcolor{teal}{#1}}}
\newcommand{\alertviolet}[1]{\textbf{\textcolor{violet}{#1}}}
\newcommand{\alertwhite}[1]{\textbf{\textcolor{white}{#1}}}
\newcommand{\alertyellow}[1]{\textbf{\textcolor{yellow}{#1}}}

% Macros for question/answer
\newcommand{\question}[1]{\alertblue{Q: #1}}
\newcommand{\answer}[1]{\alertmagenta{A: #1}}

% Common signal names for MIPS design
\newcommand{\signalname}[1]{\codeviolet{#1}}

\newcommand{\clk}{\signalname{clk}}
\newcommand{\asynnrst}{\signalname{asyn\_n\_rst}}

\newcommand{\instruction}{\signalname{instruction}}

\newcommand{\pc}{\signalname{pc}}
\newcommand{\nextpc}{\signalname{next\_pc}}


\newcommand{\sa}{\signalname{sa}}

\newcommand{\rs}{\signalname{rs}}
\newcommand{\rt}{\signalname{rt}}
\newcommand{\rd}{\signalname{rd}}
\newcommand{\Qs}{\signalname{Qs}}
\newcommand{\Qt}{\signalname{Qt}}

\newcommand{\rfwe}{\signalname{rf\_we}}

\newcommand{\aluresult}{\signalname{alu\_result}}
\newcommand{\aluop}{\signalname{alu\_op}}
\newcommand{\alua}{\signalname{alu\_a}}
\newcommand{\aluasel}{\signalname{alu\_a\_sel}}

\newcommand{\op}{\signalname{op}}
\newcommand{\func}{\signalname{func}}

% Macro for addressing modes
\newcommand{\Reg}[1]{\code{R[{#1}]}\xspace}
\newcommand{\Mem}[1]{\code{Mem[{#1}]}\xspace}
\def\R#1{\code{R#1}\xspace}

\newcommand{\Rd}{\code{Rd}\xspace}
\newcommand{\Rs}[1]{\code{Rs#1}\xspace}

%\def\Reg#1{\code{Reg[#1]}}
%\def\Mem{#1}{\code{Mem[#1]}}

% Colors for listings
\definecolor{mygreen}{rgb}{0,0.4,0}
\definecolor{mygray}{rgb}{0.5,0.5,0.5}
\definecolor{mymauve}{rgb}{0.58,0,0.82}

% Commonly used commands
\newcommand{\uA}{$\mu$A\xspace}
\newcommand{\ie}{\emph{i.e.}\xspace}
\newcommand{\eg}{\emph{e.g.}\xspace}
\newcommand{\etc}{\emph{etc}\xspace}
\newcommand{\SV}{SystemVerilog\xspace}

% MIPS instruction types
\newcommand{\Rtype}{R-Type\xspace}
\newcommand{\Itype}{I-Type\xspace}
\newcommand{\Jtype}{J-Type\xspace}

\newcommand{\fref}[1]{Figure~\ref{#1}}
\newcommand{\tref}[1]{Table~\ref{#1}}
\newcommand{\aref}[1]{Appendix~\ref{#1}}
\newcommand{\sref}[1]{Section~\ref{#1}}
\newcommand{\eref}[1]{Equation~\eqref{#1}}
\newcommand{\qref}[1]{Question~\ref{#1}}
\newcommand{\lref}[1]{Listing~\ref{#1}}
\DeclarePairedDelimiter{\ceil}{\lceil}{\rceil}

% Macro for creating each row of table for Addressing Modes
% These rows are used more than once, hence their usefulness
% Header
\def\AddressModesHeader{\hline\textbf{Mode} & \textbf{Example} & \textbf{Meaning} \\ \hline\hline}

% Immediate
\def\Immediate{Immediate &
               \code{ADD} \R4 , \code{3} &
               \R4 $\leftarrow$ \R4 + \code{3} \\ \hline}

% Register
\def\Register{Register &
              \code{ADD} \R4 , \R3 &
              \R4 $\leftarrow$ \R4 + \R3 \\ \hline}

% Absolute
\def\Absolute{Absolute (Direct)&
              \code{ADD} \R2 , \code{(100)} &
              \R2 $\leftarrow$ \R2 + \Mem{100}  \\ \hline}

% Register indirect
\def\RegisterIndirect{Register indirect &
                      \code{ADD} \R4 , \code{(\R1)} &
                      \R4 $\leftarrow$ \R4 + \Mem{\R1} \\ \hline}

% Indexed
\def\Indexed{Indexed &
             \code{ADD} \R3 , \code{(\R1+ \R2)} &
             \R3 $\leftarrow$ \R3 + \Mem{\R1+ \R2}  \\ \hline}

% Displacement
\def\Displacement{Displacement &
                  \code{ADD} \R4 , \code{100(\R1)} &
                  \R4 $\leftarrow$ \R4 + \Mem{100+\R1} \\ \hline}

% Memory indirect
\def\MemoryIndirect{Memory indirect &
                    \code{ADD} \R1 , \code{@(\R3)} &
                    \R1 $\leftarrow$ \R1 + \Mem{\Mem{\R3}}  \\ \hline}

% Autoincrement
\def\Autoincrement{\multirow{2}{*}{Autoincrement} &
                   \multirow{2}{*}{\code{ADD} \R1 , \code{(\R2)+}} &
                   \R1 $\leftarrow$ \R1 + \Mem{R2}  \\
               & & \R2 $\leftarrow$ \R2 + \emph{d}  \\ \hline}

% Autodecrement
\def\Autodecrement{\multirow{2}{*}{Autodecrement} &
                   \multirow{2}{*}{\code{ADD} \R1 , \code{-(\R2)}} &
                   \R2 $\leftarrow$ \R2 - \emph{d}  \\
               & & \R1 $\leftarrow$ \R1 + \Mem{R2}  \\ \hline}

% Scaled
\def\Scaled{\multirow{2}{*}{Scaled} &
            \multirow{2}{*}{\code{ADD} \R1 , \code{100(\R2)[\R3]}} &
            \R1 $\leftarrow$ \R1 + \code{Mem[100+\R2}  \\
           & & ~~~~~~~~~~~~+\code{ \R3*d]} \\ \hline}

% Create a slide with an example of an addressing mode
\newcommand{\AddrModeEx}[3]{
\begin{frame}{\acs{ISA} characteristics}{Addressing modes: Example}
  \begin{table}[htbp]
    \centering
      \begin{tabular}{l|l|l}
       \AddressModesHeader
       #1
  	  \end{tabular}
  \end{table}
  \begin{minipage}{0.25\textwidth}
\vspace{-44pt}
  \begin{table}[htbp]
    \begin{tabular}{r|r}
	  \textbf{Reg} & \textbf{Val}\\ \hline\hline
	  \R1 & 23 \\ \hline
	  \R2 & 11 \\ \hline
	  \R3 &  7 \\ \hline
	  \R4 & 19 \\ \hline
    \end{tabular}
  \end{table}
\end{minipage}%
\begin{minipage}{0.25\textwidth}
  \begin{flushleft}
    \begin{table}[htbp]
      \begin{tabular}{r|r}
	    \textbf{Mem} & \textbf{Val}\\ \hline\hline
	     7 & 23 \\ \hline
	    11 & 13 \\ \hline
	    13 & 31 \\ \hline
	    23 & 17 \\ \hline
	    34 & 37 \\ \hline
	   100 & 13 \\ \hline
	   123 & 29 \\ \hline
	   132 & 41 \\ \hline
      \end{tabular}
    \end{table}
  \end{flushleft}
\end{minipage} 
  \begin{minipage}{0.45\textwidth}
    \vspace{-115pt}
    \ifx&#3&%
   % #1 is empty
    \else
   $d =$ #3\\
    \fi
    #2 = ?
  \end{minipage}
\end{frame}
}
